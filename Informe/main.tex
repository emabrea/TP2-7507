\documentclass[titlepage,a4paper]{article}
\usepackage[colorlinks=true,linkcolor=black,urlcolor=blue,bookmarksopen=true]{hyperref}
\usepackage{bookmark}
\usepackage{fancyhdr}

\usepackage[utf8]{inputenc}

\usepackage{graphicx}
\usepackage{float}

\pagestyle{fancy} % Encabezado y pie de página
\fancyhf{}
\fancyhead[L]{TP1S - Julian Ferres}
\fancyhead[R]{Algoritmos y Programación III - FIUBA}
\renewcommand{\headrulewidth}{0.4pt}
\fancyfoot[C]{\thepage}
\renewcommand{\footrulewidth}{0.4pt}

\begin{document}
\begin{titlepage} % Carátula
	\hfill\includegraphics[width=6cm]{logofiuba.jpg}
    \centering
    \vfill
    \Huge \textbf{Trabajo Práctico 2 — AlgoEmpires}
    \vskip2cm
    \Large [7507/9502] Algoritmos y Programación III\\
    Curso 1 \\ % Curso 1 para el de la tarde y 2 para el de la noche
    Segundo cuatrimestre de 2018 
    \vfill
    \begin{tabular}{ | l | l | } % Datos del alumno
      \hline
	    Alumno1: & BREA, Emanuel \\ \hline
	    Número de padrón: & 99327 \\ \hline
	    Email: & ema\_brea@hotmail.com  \\ \hline
      
        \hline
        Alumno2: & FERRES, Julian \\ \hline
        Número de padrón: & 101483 \\ \hline
        Email: & julianferres@gmail.com \\ \hline
         
        \hline
        Alumno3: &  \\ \hline
        Número de padrón: & \\ \hline
        Email: &  \\ \hline
            
        \hline
        Alumno4: &  \\ \hline
        Número de padrón: &  \\ \hline
        Email: &  \\ \hline
  	\end{tabular}
    \vfill
    \vfill
\end{titlepage}

\tableofcontents % Índice general
\newpage

\section{Introducción}\label{sec:intro}

\subsection{Objetivo}

Desarrollar una aplicación de manera grupal aplicando todos los conceptos vistos en el curso, utilizando un lenguaje de tipado estático (Java) con un diseño del modelo orientado a objetos y trabajando con las técnicas de TDD e Integración Contínua.


\subsection{Consigna General}

Desarrollar la aplicación completa, incluyendo el modelo de clases e interfaz gráfica. La aplicación deberá ser acompañada por pruebas unitarias e integrales y documentación de diseño.

\subsection{Especificación de la aplicación a desarrollar}

La aplicación consiste en un juego por turnos basado en el clásico juego Age of Empires II.

\section{Supuestos}\label{sec:supuestos}
% Deberá contener explicaciones de cada uno de los supuestos que el alumno haya tenido que adoptar a partir de situaciones que no estén contempladas en la especificación.



\section{Modelo de dominio}\label{sec:modelo}


\section{Diagramas de clase}\label{sec:diagramasdeclase}


\section{Detalles de implementación}\label{sec:implementacion}
% Explicaciones sobre la implementación interna de algunas clases que consideren que puedan llegar a resultar interesantes.


\end{document}
